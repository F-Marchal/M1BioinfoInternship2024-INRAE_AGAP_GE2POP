\newglossaryentry{latex}{
    name={LaTeX},
    description={Un système de composition de documents de haute qualité utilisé pour la production de textes scientifiques et techniques.}
}

\newglossaryentry{glossaire}{
    name={Glossaire},
    description={Une liste alphabétique de termes spécialisés et de leurs définitions.}
}

\newglossaryentry{compilation}{
    name={Compilation},
    description={Le processus de conversion du code source en un document final.}
}


\newglossaryentry{synonyme}{
    name={synonyme},
    description={Désigne les sites ou les mutations qui produisent le même acide aminé. En effet, les deux \glspl{codon} différents peuvent produire le même acide aminé}
}

\newglossaryentry{codon}{
    name={codon},
    description={"triplet de \gls{nucleotide} [...] qui détermine la synthèse cellulaire des acides aminés" \cite{Robert}},
}


\newglossaryentry{autogame} {
    name={autogame},
    description={"mode de reproduction par union de gamètes provenant du même individu, observé surtout dans le règne végétal (algues, champignons…), plus rare dans le règne animal" \cite{Robert}},
}

\newglossaryentry{outgroup} {
    name={outgroup},
    description={groupe d'espèces lié mais extérieur à aux espèces dont on souhaite construire la \gls{phylogenie}},
}

% 
\newglossaryentry{hexaploide} {
    name={hexaploïde},
    description={se dit d'une cellule ou d'un organisme qui possède six jeux complets de chromosomes (6n), soit six copies de chaque chromosome}
}
 
\newglossaryentry{heterogame} {
    name={hétérogame},
    description={mode de "reproduction sexuée par deux gamètes de morphologie différente (par ex. ovule et spermatozoïde)" \cite{Robert}}
}

\newglossaryentry{brins_bien_apparies} {
    name={brins bien appariés},
    description={deux brins d'ADN (ou d'\ARN) sont bien appariés lorsque les \glspl{nucleotide} qui les composent sont correctement appariés selon la complémentarité des bases}
}

\newglossaryentry{diploide} {
    name=diploïde,
    description={"se dit d'une cellule qui possède un jeu double de chromosomes semblables" \cite{Robert}}
}

\newglossaryentry{inflorescence} {
    name=inflorescence,
    description={"mode de groupement des fleurs d'une plante, ou groupe de fleurs" \cite{LeDico}}
}

\newglossaryentry{transcriptome} {
    name=transcriptome,
    description={ensemble des molécules d'\ARN résultant de l'expression d'une partie du génome d'un tissu cellulaire ou d'un type de cellule \cite{LeDico}},
}
\newglossaryentry{transcriptomique} {
    name=transcriptomique,
    description={relatif au \gls{transcriptome}},
}

\newglossaryentry{nucleotide} {
    name=nucléotide,
    description={"constituant élémentaire des acides nucléiques (ADN et \ARN), formé par un nucléoside associé à un phosphate". \cite{Robert} (modifiée)},
}


 % sequencage
\newglossaryentry{sequencage} {
    name=séquençage,
    description={détermination de l'ordre des \gls{nucleotide} composant une molécule d'ADN ou d'\ARN},
}

\newglossaryentry{assemblage} {
    name=assemblage,
    description={utilisation des \reads pour reconstruire la séquences originale dont ils sont issus. Ce processus donne naissance à des \contigs et peut requérir une référence. On parle d'assemblage \textit{ex-nihilo} si aucune référence n'est utilisée},
}

\newglossaryentry{mapping} {
    name=mapping,
    description={alignement de \reads sur une séquence de référence. Cela permet d'identifier la position position du \gls{read} dans un génome ainsi que les différences qu'il présente par rapport à l'original (ex: \SNP)},
}

\newglossaryentry{mappeur} {
    name=mappeur,
    description={logiciel permettant de réaliser des \glspl{mapping}},
}

\newglossaryentry{phylogenie} {
    name=phylogénie,
    description={"étude de l’évolution des êtres vivants afin de déterminer leurs liens de parenté \cite{LeDico}"},
}
\newglossaryentry{read} {
    name=read,
    description={séquence de \gls{nucleotide} obtenue directement à partir du \gls{sequencage} de l'ADN ou de l'\ARN},
}
\newcommand{\reads}{\glspl{read}}

\newglossaryentry{contig} {
    name=contig,
    description={séquence de \glspl{nucleotide} obtenue à partir de l'\gls{assemblage} de \reads. Dans le cadre de ce rapport, comme nous travaillons sur des donnés \gls{transcriptomique}, un \gls{contig} correspond à un \acrfull{ORF}, donc à un gène},
}
\newcommand{\contig}{\gls{contig}}
\newcommand{\contigs}{\glspl{contig}}

\newglossaryentry{polymorphisme} {
    name=polymorphisme,
    description={présence de plusieurs allèles (version d'un gène) au sein d'une population}
}


\newglossaryentry{pipe} {
    name=pipe,
    description={ou "tube" est un "mécanisme qui permet de chaîner des processus de sorte que la sortie d'un processus (stdout) alimente directement l'entrée (stdin) du suivant." \cite{LeDico}}
}
\newcommand{\pipe}{\gls{pipe}}

\newglossaryentry{commit} {
    name=commit,
    description={opération qui enregistre un ensemble de modifications apportées au code dans l'historique du projet}
}


\newglossaryentry{substitution} {
    name=substitution,
    description={désigne, dans le contexte de ce stage, un changement entre les séquences nucleotide de deux espèces. (\cref{PrincipeTrace})},
}

\newglossaryentry{orthologue} {
    name=orthologue,
    description={désignes des séquences héritées d'un même ancêtre.},
}

\newglossaryentry{McTest} {
    name=test de McDonald–Kreitman,
    description={test utilisé pour distinguer les effets de la sélection naturelle des variations génétiques neutres}
}
\newglossaryentry{x2} {
    name=test du khi carré,
    description={test statistique permettant de déterminer si les différences observées entre les fréquences attendues et les fréquences observées sont significatives}
}

\newglossaryentry{fisherTest} {
    name=test de Fisher,
    description={test statistique permettant de déterminer si deux variables d'un tableau de contingence sont indépendantes}
}

\newglossaryentry{wrapper} {
    name=wrapper,
    description={code informatique permettant d'encapsuler un logiciel tiers dans un autre logiciel}
}

\newglossaryentry{cluster} {
    name=cluster,
    description={ensemble d'ordinateur permettant de réaliser des calculs gourmands en ressource}
}
% ----------- Type de fichiers ------------

\newglossaryentry{gz} {
    name=$.gz$,
    description={fichier compressé au format $gz$},
}
\newcommand{\gz}{"\gls{gz}"}

\newglossaryentry{bz} {
    name=$.bz2$,
    description={fichier compressé au format $.bz2$},
}
\newcommand{\bz}{"\gls{bz}"}


\newglossaryentry{fastq} {
    name=$.fastq$,
    description={Type de fichier texte permettant de stocker des séquences nulceotidiques (relatif aux \glspl{nucleotide}) et leur qualité. Contient généralement des \reads issues d'un \gls{sequencage}},
}
\newcommand{\fastq}{"\gls{fastq}"}


\newglossaryentry{pdf} {
    name=$.pdf$,
    description={type de fichier standardisé correspondant à la norme ISO 32000}
}
\newcommand{\pdf}{"\gls{pdf}"}

\newglossaryentry{png} {
    name=$.png$,
    description={format d'image numérique}
}
\newcommand{\png}{"\gls{png}"}

\newglossaryentry{readme} {
    name=$README$,
    description={fichier texte utilisé pour décrire une application aux utilisateurs}
}
\newcommand{\readme}{\gls{readme}}


\newglossaryentry{svg} {
   name=$.svg$,
    description={format d'image numérique utilisant des vecteurs pour représenter une image. Ce type de fichiers assure que l'image pourra être étendue sans perte de qualité }
}
\newcommand{\svg}{"\gls{svg}"}

\newglossaryentry{tsv} {
   name=$.tsv$,
    description={type de fichier texte représentant un tableau. Les données y sont séparées par des tabulations}
}
\newcommand{\tsv}{"\gls{tsv}"}

\newglossaryentry{pyproject.toml} {
   name=$pyproject.toml$,
    description={fichier utilisé dans les projets \gls{Python} pour préciser les prérequis et les caractéristiques d'une application.}
}

\newglossaryentry{bam} {
    name=$.bam$,
    description={fichier de cartographie d'alignement binaire. C'est le type de fichier créer à la suite d'un \gls{mapping}},
}
\newcommand{\bam}{"\gls{bam}"}

\newglossaryentry{json} {
    name=$.json$,
    description={fichier texte structuré permettant de stocker des informations},
}
\newcommand{\json}{"\gls{json}"}

\newglossaryentry{bashrc} {
    name=$.bashrc$,
    description={fichier de configuration utilisé par les environnements \gls{Bash}}
}
% --------------- Logiciels ----------------

\newglossaryentry{fastqc} {
    name=fastqc,
    description={outil d'analyse des fichiers \gls{bam}, $.sam$ et \gls{fastq}. \cite{fastqc}},
}
\newcommand{\fastqc}{\gls{fastqc}}

\newglossaryentry{GitHub} {
    name=GitHub,
    description={plate-forme en ligne permettant de centraliser des dépôts \gls{git}. \cite{github}}
}
\newcommand{\GitHub}{\gls{GitHub}}

\newglossaryentry{git} {
    name=Git,
    description={système de contrôle de version distribué qui permet de suivre les modifications d'un code source. \cite{git}}
}

\newglossaryentry{Conda} {
    name=Conda,
    description={gestionnaire de paquets et d'environnements virtuels isolés. \cite{conda}}
}
\newcommand{\Conda}{\gls{Conda}}


\newglossaryentry{Lucidchart} {
    name=Lucidchart,
    description={éditeur de diagrammes en ligne.\cite{lucidchart}}
}

\newglossaryentry{STAR} {
    name=STAR,
    description={\gls{mappeur} efficace pour aligner de l'\ARN\,sur de l'ADN \cite{star}}
}

\newglossaryentry{dNdSpNpS} {
    name=dNdSpNpS,
    description={outil permettant d'extraire les \acrshort{dn}, \acrshort{ds}, \acrshort{pn}, \acrshort{ps} de données liées au génome. \cite{dNdSpNpS}}
}

\newglossaryentry{BWA} {
    name=BWA,
    description={\gls{mappeur} efficace pour aligner de l'\ARN\,sur des références de petite taille \cite{bwa}}
}
\newcommand{\BWA}{\gls{BWA}}

\newglossaryentry{MINIMAP} {
    name=Minimap2,
    description={\gls{mappeur} efficace pour aligner de l'\ARN\,sur de l'ADN \cite{minimap2}}
}

\newglossaryentry{OnlineGant} {
    name=OnlineGant,
    description={logiciel en ligne permettant de faire des diagrammes de Gantt \cite{gantt}}
}

\newglossaryentry{SnakeMake} {
    name=SnakeMake,
    description={système de gestion de workflow permettant d'assurer une forme de reproductibilité et de portabilité \cite{snakemake} }
}
\newcommand{\SnakeMake}{\gls{SnakeMake}}


\newglossaryentry{Singularity} {
    name=Singularity,
    description={sytéme donnant la possibilité de créer des "conteneurs" permettant à une application de tourner dans un environnement virtuelle et indépendant. Ce système permet d'assurer une forme de reproductibilité et de probabilité}
}
\newcommand{\Singularity}{\gls{Singularity}}

\newglossaryentry{Python} {
    name=Python,
    description={langage de programmation grand publique \cite{python}}
}
\newcommand{\Python}{\gls{Python}}

\newglossaryentry{R} {
    name=R,
    description={langage de programmation utilisé pour les analyses statistiques \cite{r}}
}

\newglossaryentry{Bash} {
    name=Bash,
    description={langage de programmation présent par défaut sur les sytémes d'exploitation linux. \cite{bash}}
}


\newglossaryentry{WSL} {
    name=WSL,
    description={sous systéme Linux pour Windows. \cite{wsl}}
}

\newglossaryentry{Slurm} {
    name=Slurm,
    description={logiciel permettant de répartir l'exécution de programmes sur les différents noeuds d'un \gls{cluster} \cite{slurm}}
}
\newcommand{\Slurm}{\gls{Slurm}}

\newglossaryentry{GeCKO} {
    name=GeCKO,
    description={pipeline facilitant les analyses phylogénétiques (cf. \cref{sec:GeCKO}) (\cite{ardisson_gecko_2024})}
}
\newcommand{\GeCKO}{\gls{GeCKO}}


\newglossaryentry{MatPlotLib} {
    name=MatPlotLib,
    description={bibliothèque \gls{Python} permettant de générer des graphiques 2D \cite{matplotlib}}
}
\newcommand{\MatPlotLib}{\gls{MatPlotLib}}

\newglossaryentry{Pytest} {
    name=Pytest,
    description={bibliothèque \gls{Python} permettant de tester les résultat des fonctions d'un projet \cite{pytest}}
}
\newcommand{\Pytest}{\gls{Pytest}}
\newcommand{\pytest}{\gls{Pytest}}


\newcommand{\bashrc}{"\gls{bashrc}"}



\newglossaryentry{getopts} {
    name=getopt,
    description={bibliothèque \gls{Python} permettant de lire les arguments donnés à un script au moment du lancement \cite{getopt}}
}
\newcommand{\getopts}{\gls{getopts}}

\newglossaryentry{sysLib} {
    name=sys,
    description={bibliothèque \gls{Python} permettant d'accéder aux variables de l'interpréteur \gls{Python} \cite{sys}}
}

\newglossaryentry{osLib} {
    name=os,
    description={bibliothèque \gls{Python} permettant de communiquer avec le système d'exploitation \cite{os}}
}

\newglossaryentry{jsonLib} {
    name=json,
    description={bibliothèque \gls{Python} permettant d'exploiter des \gls{json} \cite{json}}
}





\newglossaryentry{Doxygen} {
    name=Doxygen,
    description={Outil de documentation automatique \cite{doxygen}}
}
\newcommand{\Doxygen}{\gls{Doxygen}}

\newglossaryentry{Latex} {
    name=\LaTeX,
    description={système de développement de documents techniques et scientifiques \cite{latex}}
}
\newcommand{\Latex}{\gls{Latex}}

\newglossaryentry{gimp} {
    name=Gimp,
    description={Outil de modification d'image \cite{gimp}}
}
\newcommand{\gimp}{\gls{gimp}}

\newglossaryentry{Seqkit} {
    name=Seqkit,
    description={outil permettant notament l'analyse de fichiers \fastq\,\cite{seqkit1} \cite{seqkit2}}
}

\newglossaryentry{PAML} {
    name=PAML,
    description={logiciel utilisé pour l'analyse phylogénétique (cf. \cref{sec:CODEML}, \cite{yang_paml_2020}) (\textit{Phylogenetic Analysis by Maximum Likelihood})}
}

\newglossaryentry{EggLib} {
    name=EggLib,
    description={bibliothèque \gls{Python} proposant des \glspl{wrapper} pour divers outils de bioinformatique et des objets permettant de manipuler des séquences biologiques (cf. \cref{sec:EggLib}) \cite{siol_egglib_2022}}
}

\newglossaryentry{CODEML} {
    name=CODEML,
    description={programme  contenu dans le package \gls{PAML} (cf. \cref{sec:CODEML}, \cite{noauthor_paml-tutorialpositive-selection_nodate})}
}

\newglossaryentry{Samtools} {
    name=Samtools,
    description={outil permettant l'analyse et la manipulation de fichiers \gls{bam}}
}



% ------------------------------- acronym -------------------------------
% ------ Others -------
\newacronym{snp}{SNP}{\gls{polymorphisme} Nucléotidique Simple}
\newcommand{\SNP}{\acrshort{snp}}
\newcommand{\SNPFULL}{\acrfull{snp}}
% ------ ------ -------


% ------ Mapping -------
\newacronym{TrEx}{TrEx}{transcriptome de référence créé par l’équipe. (\textit{Transcriptome ex-nihilo)}}
\newcommand{\TrEx}{\acrshort{TrEx}}

\newacronym{GeMo}{GeMo}{génome de référence. (\textit{Génome Moderne})}
\newcommand{\GeMo}{\acrshort{GeMo}}

\newacronym{TrMo}{TrMo}{transcriptome de référence construit à partir du génome de référence. (\textit{Transcriptome moderne})}
\newcommand{\TrMo}{\acrshort{TrMo}}

\newacronym{oldBAM}{oldBAM}{fichiers \gls{bam} générés avec \gls{BWA} en utilisant \gls{TrEx} datant d'avant le début du stage (cf. \cref{sec:donnees})}
\newcommand{\OldBam}{\acrshort{oldBAM}}

% ------ ------ -------


% \newacronym{ARN}{ARN}{molécules servant de matrice pour la synthèse des protéines (\textit{Acide RiboNucléique}}
\newcommand{\ARN}{ARN}



% ------ HeatMap -------



\newacronym{GNSPge}{GNSP$\geq$}{nombre de gènes présentant un certain nombre de \SNP}
\newcommand{\GNSPge}{\acrshort{GNSPge}}

\newacronym{GNSPeq}{GNSP$=$}{Nombre de gènes présentant au moins un certain nombre de \SNP}
\newcommand{\GNSPeq}{\acrshort{GNSPeq}}

\newacronym{NbSNP}{NbSNP}{Nombre de \SNP}
\newcommand{\NbSNP}{\acrshort{NbSNP}}

\newacronym{BamTrEx}{BamTrEx}{fichiers \gls{bam} générés avec \gls{BWA} en utilisant \gls{TrEx}}
\newcommand{\BamTrEx}{\acrshort{BamTrEx}}

\newacronym{BamTrMo}{BamTrMo}{fichiers \gls{bam} générés avec \gls{BWA} en utilisant \gls{TrMo}}
\newcommand{\BamTrMo}{\acrshort{BamTrMo}}

\newacronym{BamGeStar}{BamGeStar}{fichiers \gls{bam} générés avec \gls{STAR} en utilisant \gls{GeMo}}
\newcommand{\BamGeStar}{\acrshort{BamGeStar}}


% ------ ------- -------

% ------ Traces de séléctions -------
\newacronym{pn}{$P_n$} {nombre de sites polymorphes non \glspl{synonyme} au sein d'une population. C'est le nombre de mutations entraînant un changement dans la séquence d'acides aminés de la protéine}
\newcommand{\pn}{\acrshort{pn}}

\newacronym{ps}{$P_s$}{nombre de sites polymorphes \glspl{synonyme} au sein d'une population. C'est le nombre de mutations n'entraînant aucun changement dans la séquence d'acides aminés de la protéine}
\newcommand{\ps}{\acrshort{ps}}

\newacronym{dn}{$D_n$}{nombre de \glspl{substitution} non \glspl{synonyme} entre deux populations. C'est le nombre de sites différents entraînant un changement dans la séquence d'acides aminés de la protéine}
\newcommand{\dn}{\acrshort{dn}}

\newacronym{ds}{$D_s$}{nombre de \glspl{substitution} \glspl{synonyme} entre deux populations. C'est le nombre de sites différents n'entraînant pas de changements dans la séquence d'acides aminés de la protéine}
\newcommand{\ds}{\acrshort{ds}}


\newacronym{pnps}{$\frac{P_n}{P_s}$}{représente la balance entre les mutations \glspl{synonyme} (\ps) et non \glspl{synonyme} (\pn) dans une unique population}
\newcommand{\pnps}{\acrshort{pnps}}

\newacronym{dnds}{$\frac{D_n}{D_s}$}{aussi appelé 
\acrshort{omega} ou $\frac{K_a}{K_s}$, ce ratio représente la balance entre les \glspl{substitution} \glspl{synonyme} (\ds) 
et non \glspl{synonyme} (\dn) entre deux populations. Si $\frac{D_n}{D_s} > 1$ la substitution est probablement conservée par la sélection naturelle. A l'inverse, si $\frac{D_n}{D_s} < 1$ les substituions sont probablement éliminées par la sélection naturelle. La significativité du résultats se calcule en utilisant un \gls{x2}
}
\newcommand{\dnds}{\acrshort{dnds}}



\newacronym{omega}{\ensuremath{\omega}}{autre écriture de \dnds}
\newcommand{\w}{\acrshort{omega}}

\newacronym{ORF}{Open Reading Frame} {désigne la transcrite des gènes ou cadre ouvert de lecture)}
\newacronym{REGEX}{REGEX} {suite de caractère spécifique décrivant une liste de caractères possible. Exemple \lstinline{^re.*} correspond à tous les mots (ou toutes les lignes dépendant du contexte) qui commencent par "re"}


% ------ -------------------- -------
