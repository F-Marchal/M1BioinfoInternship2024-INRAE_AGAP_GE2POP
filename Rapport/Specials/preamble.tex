% ----------------------------  PACKAGES ---------------------------- 
\documentclass[french,a4paper,11pt]{article}
\usepackage[utf8]{inputenc}
\usepackage[T1]{fontenc}
\usepackage[french]{babel}          % Switch french with english
\usepackage{csquotes}
\usepackage{tcolorbox}
\usepackage{amsmath}
\definecolor{darkblue}{rgb}{0.0, 0.0, 0.4} % Darker shade of blue
\definecolor{darkgreen}{rgb}{0.0, 0.4, 0.0} % Darker shade of blue
\definecolor{darkred}{rgb}{0.4, 0.0, 0.0} % Darker shade of blue
\usepackage[colorlinks=true, linkcolor=darkblue, citecolor=darkgreen, filecolor=purple, urlcolor=darkred]{hyperref}  
\usepackage[capitalise,noabbrev]{cleveref} 
\usepackage{graphicx}               % For including figures
\usepackage{caption}                % For better control over captions in tables and figures
\usepackage{multicol}
\usepackage{multirow} 
\usepackage{listings}
\usepackage{subfiles}
\usepackage[backend=biber,style=alphabetic,sorting=none]{biblatex}
\usepackage{booktabs} 
\usepackage{array} % Pour définir une colonne d'alignement avec un style particulier
\usepackage[acronym]{glossaries} % nonumberlist pour effacer les numéros
\usepackage{subcaption}
\usepackage{pdflscape}
\usepackage{siunitx}
\sisetup{group-separator = \ }
\usepackage{titlesec}
\usepackage{needspace}
\usepackage{enumitem}
\usepackage{chngcntr} % Permet de changer le compteur de figure et de table


%\usepackage{helvet}
% \renewcommand{\familydefault}{\sfdefault}
% % Use Adobe Helvetica for math
% \usepackage{sansmath}
% \sansmath
% ------------------------------------------------------------------- 

% ----------------------------  ALLIASES ---------------------------- 
\newcommand{\redbracket}[1]{\textbf{\textcolor{red}{[ #1 ]}}}
\newcommand{\purplebracket}[1]{\textbf{\textcolor{purple}{[ #1 ]}}}
\newcommand{\greenbracket}[1]{\textbf{\textcolor{darkgreen}{[ #1 ]}}}
% -------------------------------------------------------------------


% ----------------------------  CODES ---------------------------- 
\definecolor{codeblue}{rgb}{0.1,0.1,0.7}
\lstset{
    language=bash,                          % The programming language of the code
    basicstyle=\ttfamily\small,             % The style of the code
    keywordstyle=\color{codeblue},          % Keyword style
    commentstyle=\color{green!50!black},    % Comment style
    numbers=left,                           % Line numbers on the left
    numberstyle=\tiny\color{gray},          % Line number style
    stepnumber=1,                           % Line numbers incrementing by 1
    showstringspaces=false,                 % Don't show spaces in strings
    tabsize=4,                              % Tab size
    breaklines=true,                        % Automatic line breaking
    breakatwhitespace=false,                % Automatic breaks at whitespace
    frame=single,                           % Frame around the code
    captionpos=b,                           % Caption position (bottom)
    morekeywords={} % Define size_t as a keyword
}
% ---------------------------------------------------------------- 


% ----------------------------  GEOMETRY ---------------------------- 
\usepackage[top=3cm, bottom=3cm, left=3cm, right=3cm, marginparwidth=2.5cm]{geometry}
% ------------------------------------------------------------------- 


% ----------------------------  SECTIONS ---------------------------- 
% Redéfinir \section
\titleformat{\section}
  {\vspace{1em}\normalfont\Large\bfseries}
  {\thesection}{1em}{}

\titlespacing*{\section}
  {0pt}{\dimexpr\baselineskip}{\dimexpr\baselineskip/2}

% Redéfinir \subsection
\titleformat{\subsection}
  {\vspace{1em}\normalfont\large\bfseries}
  {\thesubsection}{1em}{}

\titlespacing*{\subsection}
  {0pt}{\dimexpr\baselineskip}{\dimexpr\baselineskip/2}

% Redéfinir \subsubsection
\titleformat{\subsubsection}
  {\vspace{1em}\normalfont\normalsize\bfseries}
  {\thesubsubsection}{1em}{}

\titlespacing*{\subsubsection}
  {0pt}{\dimexpr\baselineskip}{\dimexpr\baselineskip/2}
% ------------------------------------------------------------------- 



% ----------------------------  ITEMIZE ---------------------------- 
% Ajouter un saut de ligne après les environnements d'énumération
\setlist[itemize]{after=\vspace{0.5\baselineskip}}
\setlist[enumerate]{after=\vspace{0.5\baselineskip}}
\setlist[description]{after=\vspace{0.5\baselineskip}}
% ------------------------------------------------------------------



% ----------------------------  CREF ---------------------------- 
% Redéfinir les noms des types de références en minuscules
\crefname{section}{section}{sections}
\crefname{figure}{figure}{figures}
\crefname{table}{tableau}{tables}

% Optionnel : Redéfinir les noms avec une majuscule (pour \Cref)
\Crefname{section}{Section}{Sections}
\Crefname{figure}{Figure}{Figures}
\Crefname{table}{Tableau}{Tables}
% -------------------------------------------------------------- 


% ----------------------------  BIBLIO ---------------------------- 
\addbibresource{ref.bib}
% ----------------------------------------------------------------- 
