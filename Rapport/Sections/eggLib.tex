% ----------------------------  START --------------------------- 
\documentclass[../main]{subfiles} % main refers to main.tex
\graphicspath{{\subfix{../Illustrations}}}
\begin{document}
\addto\extrasfrench{\protected\edef:{\unexpanded\expandafter{:}}}
\selectlanguage{french}
% ---------------------------------------------------------------- 

\section{Pipeline pour la recherche de traces de sélections}
\label{sec:PipelineTrace}
La création de cette pipeline a débuté pendant le \gls{mapping} mentionné dans la \cref{sec:Minimap} (cf. \cref{fig:gantt}). Trop peu de temps a été consacré à cette pipeline pour qu'elle soit dans un état utilisable. Cependant, le terrain a bien été débroussaillé. Cette section a pour but de présenter les concepts théoriques et les avancements de la pipeline.

\subsection{Objectif et contexte}
L'objectif de cette pipeline est de détecter automatiquement des traces de sélection sur des gènes. Pour cela la pipeline doit pouvoir accepter deux types de sources :

\begin{description}
    \item[alignements de gènes \glspl{orthologue}] Ces alignements doivent être réalisés au niveau des \glspl{codon} et non des \glspl{nucléotide}, la sélection se faisant principalement au niveau de la protéine. Lorsque ce type de fichier est fournis, une analyse utilisant \gls{CODEML} doit être réalisée.
    
    \item[fichier de statistiques] Ces fichiers tabulés doivent contenir des informations concernant le \acrshort{pn}, le \acrshort{ps}, le \acrshort{dn}, et le \acrshort{ds}. Dans ce type de situation un \gls{McTest} est réalisé.
\end{description}

Dans le cadre de ce stage, le but est d'identifier des traces de sélection sur les gènes \glspl{orthologue} identifiés chez les différents individus mentionnés dans la \cref{fig:Phylo}.

\subsection{Fonctionnement théorique}
\label{sec:PipelineFoncionnement}
Le Fonctionnement théorique du pipeline est le suivant :

\begin{itemize}
    \item Analyse avec \gls{CODEML} / \gls{EggLib}.  
    \begin{enumerate}
        \item Chargement du fichier d'alignement (\lstinline{egglib.io.from_fasta})
        \item Chargement de l'arbre ayant permis la création de l'alignement (\lstinline{egglib.Tree})
        \item Déclenchement de \gls{CODEML} sur les modèles $M1a$, $M2a$, $M7$ et $M8$  (\lstinline{egglib.wrappers.codeml}).
        \item \Gls{x2} sur les résultats : $M1a$ contre $M2a$ et $M7$ contre $M8$.
        \item Si un des tests est significatif, le résultat doit être écrit dans un fichier de sortie.
    \end{enumerate}
    
    \item Analyses avec \gls{McTest}.
        \begin{enumerate}
        \item Chargement du fichier contenant les statistiques 
        \item Pour chaque fichier : \gls{fisherTest} pour identifier si \acrshort{dnds} est significativement supérieur à \acrshort{pnps}
        \item Si le test est significatif, écriture du résultat dans un fichier de sortie.
    \end{enumerate}
\end{itemize}

Dans l'idéal pour chaque fichier traité et pour chaque modèle, le pipeline doit utiliseé du multiprocessing, du thtreading ou \gls{Slurm}.

\subsection{Les outils}

Présentons succinctement les logiciels utilisés.

\subsubsection{CODEML}
\label{sec:CODEML}
\gls{CODEML} est un outil disponible dans \gls{PAML} (Phylogenetic Analysis by Maximum Likelihood), un ensemble d'outil permettant de faire des analyses relatives aux \glspl{phylogenie}. \gls{CODEML} permet d'analyser les pressions sélectives agissant sur des séquences codantes. Dans notre cas, la "sélection positive" est définie par la présence de \glspl{codon} où \acrshort{omega} > 1  \cite{alvarez-carretero_beginners_2023}. Pour cela, plusieurs modèles sont disponibles. Dans le cadre de ce stage nous ne nous intéresserons qu'aux modèles "sites" :
\begin{itemize}
    \item $M1a$ Modèle "quasiment neutre".  \cite{alvarez-carretero_beginners_2023}
    \item $M2a$ Modèle "sélection positive". \cite{alvarez-carretero_beginners_2023}
    \item $M7$ suit une distribution Beta, ce modèle ne permet pas d'attribuer une sélection positive à un site. \cite{yang_codon-substitution_2000}
    \item $M8$ modèle similaire à $M7$, ce modèle ajoute un paramètre permettant d'attribuer une sélection positive à un site. La comparaison des résultats de ce modèle avec $M7$ permet de détecter les sites subissant une sélection positive \cite{yang_codon-substitution_2000}

\end{itemize}


\subsubsection{EggLib}
\label{sec:EggLib}
\gls{EggLib} est un package \gls{Python} proposant des objets permettant la manipulation d'objets biologiques (alignements, arbres phylogénétiques, séquences ...). En plus de ces objets, \gls{EggLib} propose des \glspl{wrapper} pour différents logiciels. Nous utiliserons principalement \gls{EggLib} pour son \gls{wrapper} \gls{CODEML}. Il permet, en théorie, d'exécuter \gls{CODEML}, de lire automatiquement le fichier de sortie et de charger les informations de celui-ci dans un dictionnaire \gls{Python}. Cela facilite grandement l'exploitation des résultats de \gls{CODEML}.

\label{sec:FamiCodeml}
Afin de me familiariser avec \gls{CODEML}, j'ai suivi un "Beginner's Guide" (\cite{alvarez-carretero_beginners_2023}). Celui-explique les différents types de tests réalisables avec cet outil et propose des données permettant de reproduire les exemples utilisés dans le guide. Celles-ci se trouvent sur un dépôt \GitHub \cite{noauthor_paml-tutorialpositive-selection_nodate}.

La reproduction des résultats s'est faite sans accrocs. Seule la partie $02\_branch\_models$ a été utilisée car elle contient les tests que nous souhaitons effectuer.


\subsection{Familiarisation avec  \gls{EggLib}}

Pour me familiarisé avec \gls{EggLib}, et plus particulièrement avec le \gls{wrapper} de  \gls{CODEML}, j'ai tenté de reproduire les manipulations faites avec \gls{CODEML} (\cref{sec:FamiCodeml}) dans un script \gls{Python}.

\subsubsection{Problèmes avec \gls{EggLib}}
Lors de ma reproduction des résultats, j'ai découvert deux problème majeurs.

\paragraph{Gestion des séquences} Lorsque la première séquence de l'alignement contient des délétions ("$---$"), la fonction \lstinline {egglib.wrappers._codeml._helper_rst} n'arrive pas à extraire les données. Celui-ci considére la délétion comme des caractères invalides.

\paragraph{Arguments du \gls{wrapper}} L'argument "codon\_freq" été limité à une valeur de 3 tandis que \gls{CODEML} peut accepter plus de valeur. A titre indicatif, \cite{alvarez-carretero_beginners_2023} utilise une valeur de 7. 

\vspace{1em}
Ces deux problèmes ont été rapportés au développeur et ont été résolus avec le passage à la version $3.3.3$.

\paragraph{\acrshort{REGEX} incomplète}
La \acrshort{REGEX} utilisée pour exploiter le fichier de sortie de \gls{CODEML} ne fonctionne pas  correctement :

\begin{enumerate}
    \item la présence de délétions ("$---$") dans l'alignement casse la \acrshort{REGEX} (le symbole "-" n'est pas reconnu par celle-ci)
    \item la "log-likelihood" qui est requise par certains tests n'est pas récupérée par le \gls{wrapper}
    \item la valeurs de $P(w)>1$ associé aux sites potentiellement sélectionnés ne sont pas récupérés.
\end{enumerate}

J'ai tenté de corriger la \textit{REGEX} (cf. \cite{florent_f-marchalm1bioinfointernship2024-inrae_agap_ge2pop_2024} $Regex$) mais je ne suis pas arrivé au bout de la correction du point n°3. 

\redbracket{Je souhaite envoyer les modifications que j'ai apporté à la REGEX à Stéphane De Mita, pensez-vous que cela soit utile / nécessaire ?}

\subsection{Conclusion}
\label{sec:EggConclusion}
Si ce stage n'a pas permis de construire cette pipeline, il aura permis de mettre en évidence des problèmes liée à \gls{EggLib}. L'identification de ces problèmes et la construction théorique de la pipeline devrait permettre dans l'avenir de reprendre la construction de la pipeline.
% ----------------------------  END --------------------------- 

\end{document}
% -------------------------------------------------------------
