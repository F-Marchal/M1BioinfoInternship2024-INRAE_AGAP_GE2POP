% ----------------------------  START --------------------------- 
\documentclass[../main]{subfiles} % main refers to main.tex
\graphicspath{{\subfix{../Illustrations}}}
\begin{document}
\addto\extrasfrench{\protected\edef:{\unexpanded\expandafter{:}}}
\selectlanguage{french}
% ---------------------------------------------------------------- 

\section{Conclusion générale}

Le jeu de données étudié risque fortement de ne pas convenir pour réaliser des recherches de traces de sélection chez des espèces apparentées au blé. En effet, comme vu dans la \cref{sec:SNP_Results}, le nombre de \SNP\,par \contig\,est trop faible. De plus, les re-\glspl{mapping} (\cref{sec:MapConclusion}) montrent que la référence (\TrEx) utilisée pour identifier les \SNP est la meilleure. 

On notera tout de même qu'il existe deux points à poursuivre pour confirmer ces résultats :

\begin{itemize}
    \item les \bam issus des \glspl{mapping} sur \GeMo n'ont pas été analysés (\cref{sec:Star}). Il est tout à fait possible que ce \gls{mapping} soit meilleur et plus complet que les précédents.

    \item les tableaux présentant le nombre de \SNP par \contig (cf. (\cref{sec:donnees})) n'ont été fait qu'à partir des \OldBam. Si les tableaux ont été générés après le filtrage drastique qu'ils ont subi (\cref{sec:oldBam}), le nombre de \SNP peu avoir été grandement affecté. Refaire ces tableaux avec les nouveaux fichiers (\BamGeStar, \BamTrEx, \BamTrMo) pourrait permettre de grandement augmenter le nombre de \SNP\,trouvés.
\end{itemize}

\subsection{Apprentissage personnel}
Enfin, ce stage à été l'occasion pour moi de développer de nouvelles compétences et d'apprendre à me servir de nouveaux logiciels. Le \cref{tab:software_usage}, en annexe, fournit une liste non exhaustive des outils que j'ai manipulés et des compétences que j'ai acquises ou commencé à acquérir.






% ----------------------------  END --------------------------- 

\end{document}
% -------------------------------------------------------------
