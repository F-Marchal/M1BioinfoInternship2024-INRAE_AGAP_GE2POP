\documentclass{beamer}
\usepackage{hyperref}

\usepackage[T1]{fontenc}

% -------------------------------------------------------------------
% other packages


% dummy text; remove it when working on this template
\usepackage{lipsum}


\date{\today}
\usepackage{./Ritsumeikan}

% defs
\def\cmd#1{\texttt{\color{red}\footnotesize $\backslash$#1}}
\def\env#1{\texttt{\color{blue}\footnotesize #1}}
\definecolor{deepblue}{rgb}{0,0,0.5}
\definecolor{ccnuMainColor}{RGB}{0,86,109}
\definecolor{deepgreen}{rgb}{0,0.5,0}
\definecolor{halfgray}{gray}{0.55}


% -------------------------------------------------------------------

% ----------------------------  PACKAGES ---------------------------- 
% \usepackage{latexsym,calligra}
% \usepackage{pstricks,stackengine}


\usepackage[utf8]{inputenc}
\usepackage[T1]{fontenc}
\usepackage[french]{babel}          % Changer 'french' avec 'english' si vous préférez l'anglais
\usepackage{csquotes}
\usepackage{graphicx}               % Pour inclure des images
\usepackage{caption}                % Pour mieux contrôler les légendes dans les tableaux et figures
\usepackage{booktabs}               % Pour les tableaux avec lignes horizontales
\usepackage{amsmath}                % Pour les équations mathématiques
\usepackage{multicol}               % Pour les environnements multi-colonnes
\usepackage{multirow}               % Pour fusionner les lignes dans les tableaux
\usepackage{listings}               % Pour afficher le code source
\usepackage{subcaption}             % Pour les sous-légendes

\usepackage{tcolorbox} 
\usepackage{subfiles}
\usepackage{tcolorbox} 

\usepackage{xcolor}
\usepackage{siunitx} % \num (affichage gros nombre)
\sisetup{group-separator = \ } % Espace insecable pour seppare les chiffres des nombre (5,00,00 devient 5 000 000)

\usepackage[backend=biber]{biblatex} 
% \usepackage[style=authoryear, backend=biber]{biblatex} % Load biblatex with the biber backend
\addbibresource{ref.bib} % Replace with your .bib file

